\documentclass[12pt, letterpaper]{article}
\usepackage[margin={1.5in}]{geometry}

\usepackage{amsmath}
\usepackage{amsthm}
\usepackage{amssymb}
\usepackage{amsfonts}

\usepackage{hyperref}
\usepackage{authblk}
\usepackage[acronym]{glossaries}
\usepackage{xcolor}

\hypersetup{
	colorlinks,
	linkcolor={red!50!black},
	citecolor={blue!50!black},
	urlcolor={blue!80!black}
}


% My commands
\newcommand{\R}{\mathbb{R}}
\newcommand{\N}{\mathbb{N}}
\newcommand{\Z}{\mathbb{Z}}
\newcommand{\T}{\mathrm{T}}
\newcommand{\E}{\mathbb{E}}
\newcommand{\Var}{\mathrm{Var}}
\newcommand{\Cov}{\mathrm{Cov}}
\newcommand{\Dcal}{\mathcal{D}}

\newtheorem{proposition}{Proposition}
\newtheorem{theorem}{Theorem}
\newtheorem{example}{Example}

\makeglossaries
\newacronym{dro}{DRO}{Distributionally Robust Optimization}
\newacronym{dddro}{DDDRO}{Decision-Dependent Distributionally Robust Optimization}

\title{
	Decision-Dependant Distributionally Robust Optimization \\
	\large Facility Location Problem with Random Demands
}
\author[1]{Gabriel Fortin-Leblanc}
\author[2]{Mohammad Joshaghani}
\affil[1]{Université de Montréal}
\affil[2]{Université du Québec à Montréal}
\date{December 15, 2024}

% Requested:
% [ ] 15-20 pages
% [x] Presentation of the decision problem in its deterministic form
% [x] Motivation regarding the key uncertainties in this problem
% [x] Presentation of a robust optimization model
% [ ] Short review of closely related litterature
% [ ] Description of the solution scheme (either through reformulation or decomposition methods)
% [ ] Presentation of experimental setup
% [ ] Summary of numerical findings
% [ ] Improvements?
% [ ] Add an abstract?

\begin{document}
	\maketitle
	\tableofcontents
	\newpage
	
	\section*{Introduction}
	Facility location is an important problem in transportation and logistics systems. It involves decisions about where to establish facilities to serve customer demands effectively while minimizing costs \cite{cornuejols_uncapicitated_1983}. Traditionally, facility location models assume demand is exogenous, fixed, or governed by a known probability distribution, but the models became more complex to better reflect reality. It has been used for multiple tasks that go further than positioning ``facilities", like positioning fire departments and their vehicles \cite{rodriguez_simulation-optimization_2021}, or even positioning testing facilities for COVID-19 \cite{liu_testing_2023}.
	
	Consider the problem with $J$ customer sites and $I$ possible location for new facilities. The decision variable is $y \in \{0, 1\}^I$ where for some $i \in [I]$, $y_i = 1$ represents the decision of opening the facility at the location assigned to the index $i$ or not opening the facility if $y_i = 0$. Moreover, other information on those possible sites are available: $O \in \R_+^I$ the opening costs, $C \in \R_+^I$ the capacities and $t \in \R_+^{I \times J}$ the transportation fees for transporting one unit of merchandise from a facility to a customer site. About those lasts, $r \in \R_+^J$ the revenue for selling one item and $\Dcal = \{\xi_k \in \R_+: k \in [K]\} \subset \R_+^K$ the support for the demand are known. To simplify the notation suppose that $\xi_k < \xi_{k+1}$ for any $k \in [K-1]$ and denote $D = \max \Dcal$. Note that the support of the demand is the same for every customer sites. The deterministic model is
	\begin{subequations} \label{eq:deterministic_problem}
		\begin{align}
			\min_{y \in \{0, 1\}^I} &\quad O^\T y + \sum_{i \in [I]} \sum_{j \in [J]} (t_{ij} - r_j) x_{ij} \\
			\text{s.t.} &\quad \sum_{j \in [J]} x_{ij} \le C_i y_i, \quad \forall i \in [I] \\ \label{eq:deterministic_problem:capacity}
			&\quad \sum_{i \in [I]} x_{ij} \le D, \quad \forall j \in [J] \\ \label{eq:deterministic_problem:demand}
			&\quad x_{ij} \ge 0, \quad \forall i \in [I], j \in [J]
		\end{align}
	\end{subequations}
	The free variable $x \in \R_+^{I \times J}$ represents the number of items sent from facilities to customer sites. The Constraint \eqref{eq:deterministic_problem:capacity} impose to not send more merchandise than the capacity of the a facility and the Constraint \eqref{eq:deterministic_problem:demand} impose to not send more merchandise to a customer site than the maximum demand.
	
	However, in many real-world tasks, customer demand or other parameters of the model is not deterministic. Parameters, demand in specific, have unknown probability distributions, and we face ambiguity set over realization of parameters or distributions that these parameters are sampled from. For example, in the formulation \eqref{eq:deterministic_problem}, the demand might be an unknown random variable. In this case, we can define an ambiguity set that covers scenarios from which random variable can takes value. Alternatively, we can define ambiguity set over distributions that model the demand.
	
	\subsection*{Litterature}
	% TODO: Review of litterature.
	% I was wainting more like a history of this topic!!
	Facility location problem is studied to addresses the question of where to establish a resource that most users can benefit from. The resource can be healthcare facility \cite{ahmadi2017survey} and police stations \cite{borba2022optimizing}.
	
	\cite{cornuejols_uncapicitated_1983} started the modeling from deterministic and continue to elaborate how to account uncertainty of parameters. In the following, we discuss papers that address the uncertainty in the problem with robust optimization approaches.

	\cite{cheng_distributionally_2024} addresses the capacitated facility location problem with uncertain facility capacity and customer demand. The objective is to minimize total costs while ensuring system reliability. It employs a \gls{dro} framework with a scenario-wise ambiguity set, reformulated into a mixed-integer linear programming model for practical solutions.

	\cite{shehadeh_distributionally_2023} addresses the mobile facility fleet-sizing, routing, and scheduling problem under time-dependent, random demand using two distributionally robust optimization (DRO) models. The models minimize fleet establishment and operational costs, incorporating risk measures (expectation or CVaR) over ambiguous demand distributions. The authors propose a decomposition-based solution and symmetry-breaking constraints, improving computational efficiency and convergence.

	The optimal decision can itself affect the uncertain parameters. In this setting, integrating decision dependent uncertainty is studied in two approaches. One considers how decisions influence the time of revealing information about uncertainty, while the second approaches, focused on how decision change the underlying distribution.

	As an example of first group, \cite{basciftci2024adaptive} addresses the challenge of balancing flexibility and decision commitment in multistage stochastic programming, where frequent decision revisions are impractical. It introduces adaptive multistage stochastic programming, a novel framework that optimally determines revision times based on the allowed flexibility, combining theoretical insights with a specialized decomposition method for computational efficiency. Through experiments on stochastic lot-sizing and generation expansion planning, the proposed approach demonstrates significant performance improvements by optimizing revision stages, achieving near-flexible outcomes even in constrained settings.

	For the second approach that described the effect of the decision on the ambiguity set, \cite{rahimiandistributionally} addresses a two-stage stochastic program with continuous recourse, where the distribution of random parameters depends on the decisions, modeled using a polyhedral ambiguity set for distributional robustness. The problem is reformulated as a nonconvex two-stage stochastic program, including cases with bilinear constraints or concave objectives, and solved using finitely-convergent, decomposition-based cutting plane algorithms. Computational results demonstrate the method’s application to joint pricing and stocking decisions in a multiproduct newsvendor problem with price-dependent demand.
	
	In the next sections we discuss in detail moment-based ambiguity and how decisions can effect parameters of this set, i.e., mean and variance of the distribution, that all possible distributions must be near to them.

	\section{Distributionally robust counterpart model}
	Under the \gls{dro} scheme, the demand is a random variable. Let $d$ be the demand of the costumer sites with $\Dcal^J$ as support. Define the uncertainty set as the set of possible distributions for $d$,
	\begin{equation} \label{eq:def_uncertainty_set}
		U \subset \left\{\pi \in [0, 1]^{J \times K}: \sum_{k \in [K]} \pi_{jk} = 1, j \in [J]\right\}.
	\end{equation}
	and to force our model to sell merchandise, define $p \in \R_+^J$ the penalty for not completing the demand of one item, such that $p_j > t_{ij}$ for any $i \in [I], j \in [J]$. The \gls{dro} problem is
	\begin{equation}\label{eq:dro_outter_problem}
		\min_{y \in \{0, 1\}^I} \left\{O^\T y + \max_{\pi \in U} \E h(y, d)\right\},
	\end{equation}
	where
	\begin{subequations} \label{eq:dro_inner_problem}
		\begin{align}
			h(y, d) = \min_{x, s} &\quad \sum_{i \in [I]} \sum_{j \in [J]} t_{ij}x_{ij} + \sum_{j \in [J]} (p_j s_j - r_j d_j) \\
			\text{s.t.} &\quad \sum_{i \in [I]} x_{ij} + s_j = d_j \quad \forall j \in [J] \\ \label{eq:dro_inner_problem:demand}
			&\quad x_{ij} \le C_i y_i \quad \forall i \in [I], j \in [J] \\
			&\quad s_i, x_{ij} \ge 0 \quad \forall i \in [I], j \in [J].
		\end{align}
	\end{subequations}
	The new free variable $s$ represents the number of item not satisfied. The Constraint \eqref{eq:dro_inner_problem:demand} imposes that $s$ is the difference between the demand from costumer sites and the satisfied demand.
	
	% TODO: Write an example for DRO.
	
	The decision of opening or not a specific facility may influence the demand. In the case of carsharing, it seems there is a positive effect from the accessibility of vehicles on the demand \cite{ciari_modeling_2014}.
	% TODO: Add more sauce to this paragraph.
	
	\subsection{Decision-dependant distributionally robust model}
	Assume from now that the decision affects the demand and from now the uncertainty set depends on the decision. The \gls{dro} problem remains the same except for the uncertainty set. It leads to the \gls{dddro} problem 
	\begin{equation}\label{eq:dddro_outter_problem}
		\min_{y \in \{0, 1\}^I} \left\{O^\T y + \max_{\pi \in U(y)} \E h(y, d)\right\}.
	\end{equation}
	Since there is a three-level optimization, our first goal is to get a single-level optimization.
	
	\begin{proposition} \label{prop:duality_inner_problem}
		The Problem \eqref{eq:dro_inner_problem} is equivalent to
		\begin{equation} \label{eq:dro_equiv_inner_problem}
			h(y, d) = \sum_{j \in [J]} \left(\max_{i' = 0, \dots, I} \left\{t_{ij} d_j + \sum_{i \in [I]: t_{ij} < t_{i'j}} C_{i}y_{i}(t_{ij} - t_{i'j})\right\} - r_j d_j\right),
		\end{equation}
		where $t_{0j} = p_j$ (Proposition 1 of \cite{basciftci_distributionally_2021}).
	\end{proposition}
	
	Remark that the Proposition \ref{prop:duality_inner_problem} is independent of $U(y)$. Basciftci et al. push further the work, but for a specific case of uncertainty set,
	\begin{equation} \label{eq:def_dd_BAS_uncertainty_set}
		\begin{split}
			U(y) = \Bigg\{
				\pi \in [0, 1]^{J \times K} : \sum_{k \in [K]} \pi_{jk} = 1, \\
				\left|\E_{\pi_j}[d] - \mu_j(y)\right| < \varepsilon_j^\mu, \\
				\left(\sigma_j^2(y) + (\mu_j(y))^2\right)\underline{\varepsilon}_j^\sigma \le
				\E_{\pi_j}[d^2] \le \left(\sigma_j^2(y) + (\mu_j(y))^2\right)\overline{\varepsilon}_j^\sigma, \\
				j \in [J]
			\Bigg\}.
		\end{split}
	\end{equation}	
	
	\begin{theorem}
		With the uncertainty set that respect the form of Equation \eqref{eq:def_dd_BAS_uncertainty_set}, the Problem \eqref{eq:dddro_outter_problem} is equivalent to the single-level minimization problem
		\begin{subequations}
			\begin{multline}
				\min_{y, \alpha, \delta^1, \delta^2, \gamma^1, \gamma^2} O^\T y + \sum_{j \in [J]} \quad \Bigg( \alpha_j + \delta_j^1 \left(\mu_j(y) + \varepsilon_j^\mu\right) - \delta_j^2 \left(\mu_j(y) - \varepsilon_j^\mu\right) \\
				+ \gamma_j^1 \left(\sigma_j^2(y) + (\mu_j(y))^2\right)\overline{\varepsilon}_j^\sigma
				- \gamma_j^2 \left(\sigma_j^2(y) + (\mu_j(y))^2\right)\underline{\varepsilon}_j^\sigma \Bigg)
			\end{multline}
			\begin{align}
				\text{s.t.}&\quad\alpha_j + (\delta_j^1 - \delta_j^2)\xi_k + (\gamma_j^1 - \gamma_j^2)\xi_k^2 \ge \theta_{jk}(y), \forall j \in [J], k \in [K], \\
				&\quad y \in \{0, 1\}^I, \quad \delta_j^1, \delta_j^2, \gamma_j^1, \gamma_j^2 \ge 0, \forall j \in [J],
			\end{align}
		\end{subequations}
		where
		\begin{equation*}
			\theta_{jk}(y) = t_{i^*_{jk}j} \xi_k + \sum_{i \in [I]: t_{ij} < t_{i^*_{jk}j}} C_{i} y_{i} (t_{ij} - t_{i^*_{jk}j}) - r_j \xi_k,
		\end{equation*}
		with
		\begin{equation*}
			i^*_{jk} = \arg\max_{i' = 0, \dots, I} t_{ij} d_j + \sum_{i \in [I]: t_{ij} < t_{i'j}} C_{i}y_{i}(t_{ij} - t_{i'j}).
		\end{equation*}
		
	\end{theorem}
	% TODO: Present the DDDR model
	% TODO: Show how to get a tractable model
	
	\section{Experiments}
	% TODO: Present experiments
	
	\section{Further research}
	% TODO: What has to be done?
	
	\clearpage
	\printglossary[type=\acronymtype]
	
	\clearpage
	\bibliography{../ref}
	\bibliographystyle{apalike}
\end{document}
