\documentclass[12pt, letterpaper]{article}
\usepackage[margin={1.5in}]{geometry}

\usepackage{amsmath}
\usepackage{amsthm}
\usepackage{amssymb}
\usepackage{amsfonts}

\usepackage{hyperref}

% My commands
\newcommand{\R}{\mathbb{R}}
\newcommand{\N}{\mathbb{N}}
\newcommand{\Z}{\mathbb{Z}}
\newcommand{\T}{\mathrm{T}}
\newcommand{\E}{\mathbb{E}}
\newcommand{\Var}{\mathrm{Var}}
\newcommand{\Cov}{\mathrm{Cov}}
\newcommand{\Dcal}{\mathcal{D}}

\newtheorem{proposition}{Proposition}

\usepackage{authblk}


\title{
	Decision-Dependant Distributionally Robust Optimization \\
	\large Facility Location Problem with Random Demands
}
\author[1]{Gabriel Fortin-Leblanc}
\author[2]{Mohammad Joshaghani}
\affil[1]{Université de Montréal}
\affil[2]{Université du Québec à Montréal}
\date{December 15, 2024}

% Requested:
% - 15-20 pages
% - Presentation of the decision problem in its deterministic form
% - Motivation regarding the key uncertainties in this problem
% - Presentation of a robust optimization model
% - Short review of closely related litterature
% - Description of the solution scheme (either through reformulation or decomposition methods)
% - Presentation of experimental setup
% - Summary of numerical findings
% - Improvements?

\begin{document}
	\maketitle
	\tableofcontents
	\newpage
	
	\section*{Introduction}
	% TODO: Introduce to the problem.
	% TODO: Why this problem is interesting? Examples with citation.
	Facility location is an important problem in transportation and logistics systems. It involves decisions about where to establish facilities to serve customer demands effectively while minimizing costs (\cite{cornuejols1983uncapicitated}). Traditionally, facility location models assume demand is exogenous, fixed, or governed by a known probability distribution.
	
	Consider the problem with $J$ costumer sites and $I$ possible location for new facilities. The decision variable is $y \in \{0, 1\}^I$ where for some $i \in [I]$, $y_i = 1$ represents the decision of opening the facility at the location assigned to the index $i$ or not opening the facility if $y_i = 0$. Moreover, other information on those possible sites are available: $O \in \R_+^I$ the opening costs, $C \in \R_+^I$ the capacities and $t \in \R_+^{I \times J}$ the transportation fees for transporting one unit of merchandise from a facility to a costumer site. About those lasts, $r \in \R_+^J$ the revenue for selling one item and $\Dcal = \{\xi_k \in \R_+: k \in [K]\} \subset \R_+^K$ the support for the demand are known. To simplify the notation suppose that $\xi_k < \xi_{k+1}$ for any $k \in [K-1]$ and denote $D = \max \Dcal$. Note that the support of the demand is the same for every costumer sites. The deterministic model is
	\begin{subequations} \label{eq:deterministic_problem}
		\begin{align}
			\min_{y \in \{0, 1\}^I} &\quad O^\T y + \sum_{i \in [I]} \sum_{j \in [J]} (t_{ij} - r_j) x_{ij} \\
			\text{s.t.} &\quad \sum_{j \in [J]} x_{ij} \le C_i y_i, \quad \forall i \in [I] \\ \label{eq:deterministic_problem:capacity}
			&\quad \sum_{i \in [I]} x_{ij} \le D, \quad \forall j \in [J] \\ \label{eq:deterministic_problem:demand}
			&\quad x_{ij} \ge 0, \quad \forall i \in [I], j \in [J]
		\end{align}
	\end{subequations}
	The free variable $x \in \R_+^{I \times J}$ represents the number of items sent from facilities to costumer sites. The Constraint \eqref{eq:deterministic_problem:capacity} impose to not send more merchandise than the capacity of the a facility and the Constraint \eqref{eq:deterministic_problem:demand} impose to not send more merchandise to a costumer site than the maximum demand.
	
	% TODO: Explain why robust optimization should be considered.
	However, in many real-world tasks, customer demand or other parameters of the model is not deterministic. Parameters, demand in specific, have unknown probability distributions, and we face ambiguity set over realization of parameters or distributions that these parameters are sampled from.

	For example, in the formulation \ref{eq:deterministic_problem}, the demand might be an unknown random variable. In this case, we can define an ambiguity set that covers scenarios from which random variable can takes value. Alternatively, we can define ambiguity set over distributions that model the demand. 
	
	\subsection*{Litterature}
	% TODO: Review of litterature.
	\cite{cheng2024distributionally} addresses the capacitated facility location problem with uncertain facility capacity and customer demand. The objective is to minimize total costs while ensuring system reliability. It employs a distributionally robust optimization (DRO) framework with a scenario-wise ambiguity set, reformulated into a mixed-integer linear programming model for practical solutions.

	\cite{liu2023testing} addresses the challenge of locating testing facilities and adjusting their capacity to meet varying demand during pandemics like COVID-19. It proposes a two-phase optimization framework: the first phase involves prepositioning strategies with Lagrangian relaxation, while the second phase dynamically adjusts capacity using adaptive allocation policies.

	\cite{shehadeh2023distributionally} addresses the mobile facility fleet-sizing, routing, and scheduling problem under time-dependent, random demand using two distributionally robust optimization (DRO) models. The models minimize fleet establishment and operational costs, incorporating risk measures (expectation or CVaR) over ambiguous demand distributions. The authors propose a decomposition-based solution and symmetry-breaking constraints, improving computational efficiency and convergence.

	\section{Distributionally robust counterpart model}
	% Finally, define $d \in \Dcal^J$ the demand of the costumer sites, and to force our model to sell merchandise define $p \in \R_+^J$ the penalty for not completing the demand of one item, such that $p_j > t_{ij}$ for any $i \in [I], j \in [J]$. 
	\begin{equation}\label{eq:dro_outterproblem}
		\min_{y \in \{0, 1\}^I} \left\{O^\T y + \max_{\pi \in U} \E h(y, d)\right\},
	\end{equation}
	
	\begin{subequations}
		\begin{align} \label{eq:dro_innerproblem}
			h(y, d) = \min_{x, s} &\sum_{i \in [I]} \sum_{j \in [J]} t_{ij}x_{ij} + \sum_{j \in [J]} (p_j s_j - r_j d_j) \\
			\text{s.t.} &\sum_{i \in [I]} x_{ij} + s_j = d_j \quad \forall j \in [J] \\
			&x_{ij} \le C_i y_i \quad \forall i \in [I], j \in [J] \\
			&s_i, x_{ij} \ge 0 \quad \forall i \in [I], j \in [J].
		\end{align}
	\end{subequations}
	% TODO: Introduce the deterministic model
	% TODO: Present artguments for using robust optimization
	
	\subsection{Decision-dependant distributionally robust model}
	% TODO: Present the DDDR model
	% TODO: Show how to get a tractable model
	
	\section{Experiments}
	% TODO: Present experiments
	
	\section{Further research}
	% TODO: What has to be done?
	
	\bibliography{../ref}
	\bibliographystyle{apalike}
\end{document}